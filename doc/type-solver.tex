% !TEX TS-program = pdflatex
% !TeX spellcheck = en_US
% !TEX root = uigen.tex
\documentclass{article}

\usepackage{graphicx}
%\usepackage{tikz}
%\usetikzlibrary{shapes.geometric,backgrounds,
%  positioning-plus,node-families,calc}
\usepackage{xcolor}
\usepackage{amsfonts}
\usepackage{mathtools}
\usepackage{hyperref}
\usepackage{marginnote}
\usepackage{verbatim}
\usepackage{listings}
\usepackage{amsmath,mathrsfs,amssymb}

\newcommand{\term}[1]{\mbox{\texttt{\textbf{#1}}}}
\newcommand{\run}[2]{\term{run}^{#1}\,\left[#2\right]}
\newcommand{\todo}[1]{{\bf\color{red}#1}}
\newcommand{\rel}[3]{{#1}\xrightarrow{#2}{#3}}
\newcommand{\prg}[1]{\mbox{\lstinline|#1|}}
\newcommand{\precprec}{\prec\mathrel{\mkern-5mu}\prec}

\newcommand{\grc}[2]{{#1}\,\left<{#2}\right>}
\newcommand{\java}[1]{\texttt{#1}}
\newcommand{\primi}[1]{\mathbf{#1}}
\newcommand{\cc}[1]{\lfloor{#1}\rfloor}
\sloppy

\begin{document}

\title{Java Generics Type Solver}

\author{Dmitri Boulytchev}
%{St.Petersburg State University \\ 
%  Saint-Petersburg, Russia }
%{$\mathtt{Dmitrii.Kosarev@protonmail.ch}$ \and $\mathtt{dboulytchev@math.spbu.ru}$}

\maketitle

\begin{abstract}
  This is a zero-iteration analysis for the problem of solving systems of subtyping
  equations in the Java generics type system. More specifically, we put the
  problem in the context of relational verifier-to-solver approach, which brings in a certain
  specificity to the analysis. We reiterate on the Java generics type system, subtyping
  relation and relevant notions, discuss the pecularities of relational verifier-to-solver
  approach application, and describe a prototype of functional subtyping verifier.
\end{abstract}

\section{Java Generics}

Here we briefly describe the relevant (at the first glance) subpart of Java type system. All
information in this section is directly derived from the Java Language Specification~\cite{JLS};
however the accuracy of this derivation has yet to be scrutineed. We also refrain from
reiterating on the non-generic part of Java type system assuming reader's familiarity with the
concepts of reference types, arrays, classes, interfaces, and inheritance.

The subsystem we are dealing with contains generic class/interface types, array types,
type variables and intersection types:

\[
\begin{array}{rcll}
  \mathscr{T} & = & \alpha^\mathscr{T}_\mathscr{T_\bot}&\mbox{--- type variable}\\[2mm]
              &   & \bigcap\,\mathscr{T}           &\mbox{--- intersection type}\\[2mm]
              &   & \grc{C}{\mathscr{T}^*}         &\mbox{--- class or interface type}\\[2mm]
              &   & \mathscr{T}\mbox{\java{[]}}    &\mbox{--- array type}\\[2mm]
              &   & \primi{null}                   &\mbox{--- null type}\\[2mm]
\end{array}
\]

Here $\mathscr{T}_\bot$ denotes optional type.

Type variables are bounded in the sense that they always have a certain \emph{upper bound}. This upper
bound can be either specified explicitly or assumed to be \java{java.lang.Object}. In addition
to the upper bound a type variable can be equipped with a \emph{lower bound}. The lower bound can only
be derived implicitly as a result of \emph{capture conversion}. We denote the upper bound of a variable
by  superscript and lower~--- by subscript.

Generic class (or interface) is a class/interface which is equipped with a number of
type parameters. In generic class declaration these parameters can only be type
variables with specified upper bounds. Additionally a number of \emph{direct supertypes} (notation: ``$\prec$'')
can be specified for a generic class/interface:

\[
\grc{C}{\alpha_1^{U_1}\dots \alpha_k^{U_k}}\prec\grc{S_1}{T^1_1\dots T^1_{n_1}}\dots\grc{S_m}{T^m_1\dots T^m_{n_m}}
\]

Here $C$ is the class/interface being declared, $\alpha_i^{U_i}$~--- its generic parameters with upper
bounds, $S_i$~--- its direct supertypes, among which only one type is a class type, $T^j_l$~--- type
parameters for direct supertypes, which may contain the occurences of $\alpha_i$; note: neither of $T^j_l$ can be
an intersection type.

Type variables in scopes of their declarations behave like regilar types; the scope of a type variable is
determined by the position of its introduction (either generic class/interface or generic method). It
is unclear what is the scope of type variables introduced implicitly as a result of capture
conversion (but this might be irrelevant to the problem we are dealing with).

Intersection types can not be declared explicitly; the only position in which they can be specified is
upper bounds of type parameters in generic class/interface declarations. However intersection types can be derived
as a result of capture conversion.

Besides regular types generic classes can be applied to \emph{wildcard types}. A wildcard is an anonymous
type variable (notation: ``\java{?}'') equipped with optional lower or upper bound which can not be an intersection type.
If no upper bound is specified then \java{java.lang.Object} is implicitly assumed. It is important that wildcards can not 
be used to parameterize direct supertypes in generic class/interface declarations. Array types can not be directly
parameterized by wildcards.

To establish the subtyping relation (denotation: ``$\precprec$'') between types parameterized by wildcards two additional
notions are introduces.

The first one is the ``contains'' relation ``$\supseteq$'' between wildcard types:

\[
\begin{array}{rclcl}
  \java{?}^T&\supseteq&\java{?}^S&,&T\precprec S \\[2mm]
  \java{?}^T&\supseteq&\java{?}&& \\[2mm]
  \java{?}_T&\supseteq&\java{?}_S&,&S\precprec T\\[2mm]
  \java{?}_T&\supseteq&\java{?}&& \\[2mm]
  \java{?}_T&\supseteq&\java{?}^\java{java.lang.Object}&& \\[2mm]
  T&\supseteq& T&&\\[2mm]
  T&\supseteq&\java{?}^T&&\\[2mm]
  T&\supseteq&\java{?}_T&&
\end{array}
\]

The second is capture conversion which is defined as follows. Let

\[
\grc{C}{T_1\dots T_k}
\]

be a generic class/interface type where some of $T_i$ are wildcards. Let

\[
\beta_i^{U_i}
\]

be the $i$-th type parameter of $C$'s declaration. Then capture conversion of
$\grc{C}{T_1\dots T_K}$ is the type $\grc{C}{\cc{T_1}\dots \cc{T_k}}$ where the transformation
``$\cc{\bullet}$'' is defined as follows:

\[
\cc{T_i}=\left\{
\begin{array}{lll}  
  \alpha^{U_i[\beta_j\gets \cc{T_j}]}_\primi{null}      &\mbox{if }        T_i=\java{?}  &\mbox{($\alpha$ is a fresh type variable)}\\[5mm]
  \alpha^{B\cap U_i[\beta_j\gets \cc{T_j}]}_\primi{null} &\mbox{if }        T_i=\java{?}^B &\mbox{($\alpha$ is a fresh type variable)}\\[5mm]
  \alpha^{U_i[\beta_j\gets \cc{T_j}]}_B               &\mbox{if }        T_i=\java{?}_B &\mbox{($\alpha$ is a fresh type variable)}\\[5mm]
  T_i                                           &\mbox{otherwise}  & 
\end{array}\right.
\]

Here ``$[\bullet\gets\bullet]$'' denotes a (simultaneous) substitution of type variables with types. Note, in capture conversion
type parameters of the class under conversion are substituted with the results of the conversion in all bounds.

The subtyping relation ``$\precprec$'', which is the principle notion for the problem we are dealing with, is defined as a reflexive-transitive
closure of the direct supertype relation ``$\prec$''. We already described one case of direct supertyping, the others are as follows:

\[
\begin{array}{rclcl}
  I & \prec & \java{java.lang.Object} &,& \mbox{($I$ is an interface with no}\\
    &       &                         & &\mbox{direct superinterface)}\\[2mm]
  \grc{C}{T_1\dots T_k} & \prec & \grc{C}{S_1\dots S_k} &,& \forall i\, .\, S_i\supseteq T_i\\[2mm]
  \grc{C}{T_1\dots T_k} & \prec & S &,& \grc{C}{\cc{T_1}\dots \cc{T_k}}\prec S\\[2mm]
  \bigcap T_i & \prec & T_i && \\[2mm]
  \alpha^{\bigcap T_i} & \prec & T_i &&\\[2mm]
  T & \prec & \alpha_T && \\[2mm]
  \primi{null} & \prec & T &&\\[2mm]
  T\java{[]} & \prec & S\java{[]} &,& T\prec S\\[2mm]
  \java{java.lang.Object[]} & \prec & \java{java.lang.Object} &&  \\[2mm]
  \java{java.lang.Object[]} & \prec & \java{java.lang.Cloneabe} && \\[2mm]
  \java{java.lang.Object[]} & \prec & \java{java.io.Serializable} &&
\end{array}
\]

Note: the notions ``$\supseteq$'', ``$\prec$'', and ``$\precprec$'' are defined mutually recursive: ``$\precprec$'' is used
to define ``$\supseteq$'', which is used to define ``$\prec$'', which, in turn, is the basic relation to define ``$\precprec$''.

\section{On Functional/Relational Subtyping\\ Verifier/Solver}

The approach of verifier-to-solver conversion relies on the implementation of functional verifier for the problem in question. In our
case such verifier should test if two given ground types are in the subtyping relation. This, in particular, requires implementation
of capture conversion, ``contains'' relation, and direct subtyping verifier. Then a refexive-transitive closure of the latter has to
be implemented. The functional verifier is then converted into relational form which by construction delivers a subtyping solver for
non-ground types with free variables.

A simple observation, however, makes its obvious that it is in fact much easier to implement reflexive-transitive closure directly
in relational language. Indeed, given relation $R$ its reflexive-transitive $R^*$ closure can be expressed by just

\[
R^*\,(x,\, y) = R\, (x,\, y)\vee x\equiv y\vee\exists\, z\,.\,R\,(x,\,z)\wedge R\,(z,\,y)
\]

This is rather expected by the very nature of relational programming.

However, from the implementation standpoint the application of this technique imposes a certain problem since ``$\prec$'' and ``$\precprec$''
are mutually recursive, and we expect ``$\prec$'' to be obtained as a result of relational conversion of functional implementation. In current
prototype implementation we used open recursion and the knot is tied on the very top level. We hope that this (obvious) solution will be
scalable enough to be applied in the final functional-relational mixture.


\begin{thebibliography}{99}
\bibitem{JLS}
  James Gosling \emph{et al.} The Java Language Specification. \emph{Java SE 19 Edition}, 2022-08-31 //
  \url{https://docs.oracle.com/javase/specs/jls/se19/jls19.pdf}
\end{thebibliography}

\end{document}

